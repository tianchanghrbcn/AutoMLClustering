\documentclass[8pt,twocolumn]{article} % 8pt 字体大小,双栏布局
\usepackage{ctex}          % 支持中文
\usepackage{amsmath,amssymb,amsfonts}
\usepackage{graphicx}
\usepackage{booktabs}
\usepackage{geometry}
\usepackage{setspace}
\usepackage{titlesec}
\usepackage[ruled,vlined]{algorithm2e} % 导入 algorithm2e 宏包

% 页面布局设置
\geometry{a4paper, left=0.5in, right=0.5in, top=0.5in, bottom=0.5in} % 减小页边距
\setlength{\columnsep}{0.5in} % 双栏之间的间距

% 标题格式设置
\titleformat{\section}{\large\bfseries}{\thesection\ }{0em}{} % 一级标题无点
\titleformat{\subsection}{\normalsize\bfseries}{\thesubsection\ }{0em}{} % 二级标题无点
\titleformat{\subsubsection}{\normalsize\bfseries}{\thesubsubsection\ }{0em}{} % 三级标题无点

% 行距设置
\renewcommand{\baselinestretch}{1.0} % 行距为 1 倍

% 公式编号统一格式
\numberwithin{equation}{section}

\begin{document}

%---------------------------------
%标题和作者信息
%---------------------------------
\title{\textbf{标题示例: AutoML for Clustering}}
\author{常添 \\ 所属单位 \\ \texttt{email@example.com}}
\date{\today}

%---------------------------------
% 生成标题
%---------------------------------
\maketitle

%---------------------------------
%摘要(此处根据需要添加或省略)
%---------------------------------
\begin{abstract}
摘要内容的示例。
\end{abstract}

%---------------------------------
% 第一节:引言(此处根据需要添加或省略)
%---------------------------------
\section{引言}
引言内容的示例。

%---------------------------------
% 第二章:聚类问题定义与优化目标
%---------------------------------
\section{聚类问题定义与优化目标}

以下是本章节所定义的符号与描述:

\begin{table}[ht]
\centering
\caption{符号与描述1}
\label{tab:symbols-basic}
\begin{tabular}{ll}
\toprule
\textbf{符号} & \textbf{描述} \\
\midrule
$D$ & 数据集的通用表示,包含特定场景下的数据 \\
$\mathbf{x}(D)$ & 数据集 $D$ 的特征向量,包含质量与规模信息 \\
$\mathcal{C}$ & 数据清洗方法的集合 \\
$\mathcal{H}$ & 聚类算法的集合 \\
$\mathcal{P}$ & 聚类算法的超参数空间 \\
$\Omega$ & 聚类策略的初始搜索空间 \\
$\omega$ & 聚类策略的策略组合 \\
$S(D, \omega)$ & 聚类策略 $\omega$ 在数据集 $D$ 上的综合得分 \\
$T_{\text{original}}(D)$ & 在初始搜索空间上评估的耗时 \\
$T_{\text{reduced}}(D)$ & 在优化后搜索空间上评估的耗时 \\
$\eta(D)$ & 损失率,表示优化后综合得分的平均下降比例 \\
$\mathcal{A}(D)$ & 综合加速比,表示搜索效率的综合提升程度 \\

\bottomrule
\end{tabular}
\end{table}
在本文中,我们将数据集记为 \(D\),其特征向量为 \(\mathbf{x}(D)\)。给定数据清洗方法集合 \(\mathcal{C}\)、聚类算法集合 \(\mathcal{H}\) 以及超参数空间 \(\mathcal{P}\),定义\textbf{聚类策略}为三元组合:
\begin{equation}\label{eq:omega-def}
\omega = (c, h, \boldsymbol{\theta}), 
\quad 
c \in \mathcal{C},\; h \in \mathcal{H},\; \boldsymbol{\theta} \in \mathcal{P}.
\end{equation}
所有可行组合构成\textbf{初始搜索空间}:
\begin{equation}\label{eq:search-space}
\Omega = \mathcal{C} \;\times\; \mathcal{H} \;\times\; \mathcal{P}.
\end{equation}

当 \(\Omega\) 规模非常大时,若\textbf{无法遍历}整个空间,可以采取\textbf{随机采样}或\textbf{分层采样}等方法,从 \(\Omega\) 中选取若干代表性策略 \(\omega\) 用于计算并估计综合得分,以平衡搜索精度与时间成本。

%---------------------------------
\subsection{数据集特征}
\label{subsec:data-feature}
在真实世界的聚类任务中,数据集往往同时面临多种质量问题(错误值、缺失值、噪声)。为便于对不同数据集进行横向对比与后续建模,本研究对每个数据集 \(D\) 抽取如下\textbf{特征向量}:
\begin{equation}\label{eq:feature-vector}
\mathbf{x}(D) 
= \bigl(\mathrm{ErrorRate}(D),\,
\mathrm{MissingRate}(D),\,
\mathrm{NoiseRate}(D),\,
m,\,
n\bigr),
\end{equation}
其中:
\begin{itemize}
    \item \(\mathrm{ErrorRate}(D)\):错误值占总单元的比例;
    \item \(\mathrm{MissingRate}(D)\):缺失值占总单元的比例;
    \item \(\mathrm{NoiseRate}(D)\):噪声或离群点占总单元的比例;
    \item \(m\):特征维度(属性数量);
    \item \(n\):记录条数(样本规模)。
\end{itemize}

%---------------------------------
\subsection{聚类评价指标}
\label{subsec:clustering-metrics}

给定数据集 \(D\) 与聚类策略 \(\omega\),我们选用以下评价指标:

\paragraph{Davies-Bouldin (DB) Score} 
衡量簇内紧凑度与簇间分离度,值越低越好:
\begin{equation}\label{eq:db-score}
DB(D,\omega)
= \frac{1}{K}\sum_{i=1}^{K}
\max_{j\neq i}
\biggl(\frac{S_i + S_j}{d_{ij}}\biggr),
\end{equation}
其中 \(K\) 为聚类数,\(S_i\) 表示第 \(i\) 个簇的平均离散度,\(d_{ij}\) 表示簇间中心距离。

\paragraph{Silhouette Score (轮廓系数)} 
衡量每个样本在所属簇的凝聚力与最近簇的分离度,值越高越好:
\begin{equation}\label{eq:silhouette}
\mathrm{Sil}(x)
= \frac{b(x) - a(x)}{\max\{a(x),\,b(x)\}},
\end{equation}
其中 \(a(x)\) 为 \(x\) 到同簇其他样本的平均距离,\(b(x)\) 为 \(x\) 到最近簇的平均距离。

\paragraph{综合得分}
本研究将二者线性组合得到:
\begin{equation}\label{eq:S-score}
S(D,\omega)
= \alpha\cdot \bigl(-DB(D,\omega)\bigr)
+ \beta\cdot \mathrm{Sil}(D,\omega),
\end{equation}
其中 \(\alpha,\beta>0\) 为加权系数。

%---------------------------------
\subsection{优化目标与衡量指标}
\label{subsec:optimization-and-evaluation}

当对搜索空间 \(\Omega\) 做全面或抽样评估后,我们希望找到:
\begin{equation}\label{eq:optimal-omega}
\omega^*(D) 
= \arg\max_{\omega \,\in\, \Omega} \;S(D,\omega),
\end{equation}
但在 \(\Omega\) 很大时,穷尽搜索会带来极高的时间成本。因此核心目标是:\textbf{在不显著牺牲聚类质量的前提下,尽可能减少实际运行时间}。为此,我们定义了以下两个指标:

\subsubsection{损失率}
\label{subsubsec:loss-rate}
记 \(S(D, \omega)\) 为策略 \(\omega\) 在 \(D\) 上的综合得分;令 \(\Omega'(D)\) 表示优选子空间。定义\textbf{损失率}:
\begin{equation}\label{eq:loss-rate}
\eta(D)
= 1 - \frac{\frac{1}{|\Omega'(D)|} \sum_{\omega \in \Omega'(D)} S(D, \omega)}
{\frac{1}{|\Omega|} \sum_{\omega \in \Omega} S(D, \omega)}.
\end{equation}
损失率 \(\eta(D) \in [0,1]\),越接近 0 表示优化后的平均聚类质量越接近完整搜索。

\subsubsection{综合加速比}
\label{subsubsec:acc-ratio}
评估方案 \(\omega\) 的时间耗时记为 \(T(\omega, D)\)。若在原始空间 \(\Omega\) 上做全量搜索,时间为
\begin{equation}\label{eq:T-original}
T_{\text{original}}(D) 
= \sum_{\omega \,\in\, \Omega} T(\omega, D).
\end{equation}
在优选子空间 \(\Omega'(D)\) 上搜索时为
\begin{equation}\label{eq:T-reduced}
T_{\text{reduced}}(D)
= \sum_{\omega \,\in\, \Omega'(D)} T(\omega, D).
\end{equation}
\textbf{综合加速比}定义为:
\begin{equation}\label{eq:comprehensive-acceleration}
\mathcal{A}(D)
= \bigl(1 - \eta(D)\bigr) \frac{T_{\text{original}}(D)}{T_{\text{reduced}}(D)}
\end{equation}
以综合衡量\textbf{聚类质量损失}与\textbf{评估时间降低}的平衡性。

%---------------------------------
% 第三章:先验数据与映射构建
%---------------------------------
\section{先验数据与映射构建}
为提升聚类策略搜索的效率,我们可将数据集划分为先验数据(离线学习)与测试数据(在线应用),并通过多标签学习构建“数据特征 \(\to\) 优选子空间”的映射。
以下是本章节所定义的符号与描述:

\begin{table}[ht]
\centering
\caption{符号与描述2}
\label{tab:symbols-advanced}
\begin{tabular}{ll}
\toprule
\textbf{符号} & \textbf{描述} \\
\midrule
$D_{\text{train}}$ & 先验数据集(训练集),用于离线评估和学习先验知识 \\
$D_{\text{test}}$ & 测试数据集,用于实际部署和快速优化 \\
$K$ & Top-K 大小,表示在先验阶段选取的前 $K$ 个最优方案 \\
$\mathbf{M}^{(i)}$ & 数据集 $D^{(i)}$ 的 Top-K 策略矩阵 \\
$\ell$ & 标签,表示某一优选方案的标识符 \\
$\mathcal{L}$ & 标签空间,包含所有优选方案的标签集合 \\
$\mathbf{L}^{(i)}$ & 数据集 $D^{(i)}$ 对应的多标签集合 \\
$\mathcal{M}$ & 训练集,包含所有先验数据的特征与标签集合 \\
$\mathcal{F}$ & 多标签分类器,用于预测优选方案标签 \\
$q^{(j)}$ & 标签 $\ell_{\omega^{(j)}}$ 为优选方案的概率 \\
$r$ & 预测阶段保留的最高优选标签数 \\
$\mathbf{L}'$ & 预测阶段保留的最高优选标签集合。 \\
$\Omega'(D)$ & 数据集 $D$ 的优选子空间,$\Omega'(D) \subseteq \Omega$。 \\
$G$ & 映射函数,将数据集特征向量映射到优选子空间 \\
$\hat{\omega}$ & 最优方案,即在 $\Omega'(D_{\text{test}})$ 中得分最高的组合 \\
\bottomrule
\end{tabular}
\end{table}

\subsection{先验数据集与测试数据集}
\label{subsec:Dtrain-Dtest}

\begin{itemize}
    \item \textbf{先验数据集} \(D_{\text{train}}\):包含若干历史数据集 \(\{D^{(1)}, D^{(2)}, \dots\}\),可在上面对 \(\Omega\) 进行大范围或抽样评估,形成“先验知识”。
    \item \textbf{测试数据集} \(D_{\text{test}}\):实际部署场景下的新数据集。目标是\textbf{利用先验知识},减少搜索规模并\textbf{降低评估时间}。
\end{itemize}

\label{subsec:topK-matrix}

在先验数据集 \(D^{(i)}\) 上,遍历或采样若干 \(\omega \in \Omega\),计算综合得分 \(S(D^{(i)}, \omega)\);选取\textbf{评分最高}的 \(K\) 个组合构成\textbf{Top-K 方案矩阵}
\begin{equation}\label{eq:topK-matrix}
\mathbf{M}^{(i)} = 
\begin{pmatrix}
c_1 & h_1 & \boldsymbol{\theta}_1 & S_1 \\
\vdots & \vdots & \vdots & \vdots \\
c_K & h_K & \boldsymbol{\theta}_K & S_K
\end{pmatrix},
\end{equation}
其中第 \(j\) 行的策略可记为 \(\omega_j^{(i)} = (c_j, h_j, \boldsymbol{\theta}_j)\),得分为 \(S_j\)。行从上到下按 \(S_j\) 降序排列。

%---------------------------------
\subsection{基于多标签学习的映射策略}
\label{subsec:mapping-classification}

当每个先验数据集 \(D^{(i)}\) 可对应多个\textbf{优选方案}时,本研究采用\textbf{多标签学习}来构建分类器 \(\mathcal{F}\),并基于该分类器得到映射函数 \(G\)。下文说明标签空间定义、数据构造与预测流程。

\subsubsection{标签空间与多标签分配}

将所有出现过的优选策略记为
\[
\{\omega^{(1)}, \omega^{(2)}, \ldots, \omega^{(m)}\},
\]
为每个优选策略 \(\omega^{(j)}\) 分配唯一标签 \(\ell_{\omega^{(j)}}\),形成离散\textbf{标签空间}:
\begin{equation}\label{eq:label-space}
\mathcal{L}
= \{\ell_{\omega^{(1)}},\ell_{\omega^{(2)}},\ldots,\ell_{\omega^{(m)}}\}.
\end{equation}
若先验数据集 \(D^{(i)}\) 的 Top-K 组合为
\[
\mathbf{M}^{(i)} 
= \{\omega_1^{(i)},\omega_2^{(i)},\dots,\omega_K^{(i)}\},
\]
则其\textbf{多标签}集合为
\begin{equation}\label{eq:label-space for D}
\mathbf{L}^{(i)} 
= \{\ell_{\omega_1^{(i)}}, \ell_{\omega_2^{(i)}}, \dots, \ell_{\omega_K^{(i)}}\}.
\end{equation}
可见标签 \(\ell_{\omega}\) 与聚类策略 \(\omega\) 是\textbf{一一对应}的,以便后续分类器的输出可映射回具体策略。

\subsubsection{训练数据与多标签分类器}
\paragraph{训练数据构造}
将每个先验数据集 \(D^{(i)}\) 视为一条多标签样本:
\[
(\mathbf{x}(D^{(i)}),\, \mathbf{L}^{(i)}).
\]
汇总所有先验数据集,得到训练集
\begin{equation}\label{eq:training set}
\mathcal{M}
= \{(\mathbf{x}(D^{(1)}), \mathbf{L}^{(1)}), \dots, (\mathbf{x}(D^{(N)}), \mathbf{L}^{(N)})\}.
\end{equation}
\paragraph{模型训练}
记 \(\mathcal{F}\) 为多标签分类器,它输出对于每个标签 \(\ell_{\omega^{(j)}}\) 的概率 \(q^{(j)} \in [0,1]\)。可根据具体需求使用 Binary Relevance、ML-kNN、神经网络等多标签算法。

\subsubsection{预测与映射函数 \(G\)}
在测试阶段,给定新数据集 \(D_{\text{test}}\) 的特征 \(\mathbf{x}(D_{\text{test}})\),可得到:
\begin{equation}\label{eq:classifier}
\mathcal{F}\bigl(\mathbf{x}(D_{\text{test}})\bigr)
= \{(\ell_{\omega^{(1)}}, q^{(1)}), \dots, (\ell_{\omega^{(m)}}, q^{(m)})\},
\end{equation}
从中选取概率最高的 \(r\) 个标签:
\begin{equation}\label{eq:predicted label space for test}
\mathbf{L}' 
= \{\ell_{\omega^{(j)}} \mid q^{(j)} \text{ 属于前}r\text{ 大}\},
\end{equation}
然后再映射回相应的策略,得到优化后的搜索空间:
\begin{equation}\label{eq:optimized space}
\Omega'(D_{\text{test}})
= \{\omega^{(j)} \mid \ell_{\omega^{(j)}} \in \mathbf{L}'\}.
\end{equation}
基于上述预测过程,可以将“多标签分类器”的输出转化为“映射函数”:
\begin{equation}
G\bigl(\mathbf{x}(D)\bigr)
= \Omega'(D).
\end{equation}
在测试阶段,只需在 $\Omega'(D)$ 内对相对少量的组合做聚类评估,从而显著降低评估成本并提升实际搜索速度。
%---------------------------------
% 自动化聚类优化流程
%---------------------------------
\section{自动化聚类优化流程}
\label{sec:autoML-pipeline}

\subsection{流程概念图(占位)}
\label{subsec:concept-figure}

基于上述思想,本文提出的自动化聚类优化方法主要分为训练阶段和测试阶段,下面是这两个阶段算法的伪代码实现。

\subsection{训练阶段}
\begin{algorithm}[ht]
\caption{训练阶段:生成训练数据与训练多标签分类器}
\label{alg:train-phase}
\KwIn{
    先验数据集 $D_{\text{train}}=\{D^{(1)},\dots,D^{(N)}\}$;\\
    搜索空间 $\Omega$;\\
    Top-K 大小 $K$。
}
\KwOut{多标签分类器 $\mathcal{F}$}

\SetKwFunction{GenerateTrainingData}{GenerateTrainingData}
\SetKwFunction{TrainClassifier}{TrainClassifier}

$\mathcal{M} \leftarrow \GenerateTrainingData(D_{\text{train}}, \Omega, K)$\;
$\mathcal{F} \leftarrow \TrainClassifier(\mathcal{M})$\;
\KwRet{$\mathcal{F}$}

\bigskip

\SetKwProg{Fn}{Function}{:}{}
\Fn{\GenerateTrainingData{$D_{\text{train}}, \Omega, K$}}{
  $\mathcal{M} \leftarrow \emptyset$\;
  \For{$i \leftarrow 1$ \KwTo $|D_{\text{train}}|$}{
    \ForEach{$\omega \in \Omega$ \textbf{(或采样自 $\Omega$)}}{
      计算 $S(D^{(i)}, \omega)$\;
    }
    选出 Top-K 策略 $\mathbf{M}^{(i)} = \{\omega_1^{(i)}, \dots, \omega_K^{(i)}\}$ 按得分降序\;
    映射为多标签集合 $\mathbf{L}^{(i)} = \{\ell_{\omega_1^{(i)}}, \dots, \ell_{\omega_K^{(i)}}\}$\;
    $\mathcal{M} \leftarrow \mathcal{M} \cup \{(\mathbf{x}(D^{(i)}), \mathbf{L}^{(i)})\}$\;
  }
  \KwRet{$\mathcal{M}$}
}

\Fn{\TrainClassifier{$\mathcal{M}$}}{
  \tcp{可根据具体多标签算法实现}
  训练多标签分类器 $\mathcal{F}$\;
  \KwRet{$\mathcal{F}$}
}
\end{algorithm}

\subsection{测试阶段}
\begin{algorithm}[ht]
\caption{测试阶段:寻找最优方案 \(\hat{\omega}\)}
\label{alg:test-phase}
\KwIn{
    测试数据集 $D_{\text{test}}$;\\
    多标签分类器 $\mathcal{F}$;\\
    搜索空间 $\Omega$;\\
    保留标签数 $r$。
}
\KwOut{最优方案 $\hat{\omega}$}

计算 $\mathbf{x}(D_{\text{test}})$\;
$\mathbf{L}' \leftarrow \{\}$\;
\ForEach{$\ell \in \mathcal{L}$}{
  $q_{\ell} \leftarrow \text{置信度}(\mathcal{F}, \mathbf{x}(D_{\text{test}}), \ell)$\;
  $\mathbf{L}' \leftarrow \mathbf{L}' \cup \{(\ell, q_{\ell})\}$\;
}
选取置信度最高的 $r$ 个标签 $\mathbf{L}'_{\mathrm{top}}$\;
映射回优选子空间 $\Omega'(D_{\text{test}})$\;
\ForEach{$\omega \in \Omega'(D_{\text{test}})$}{
    计算 $S(D_{\text{test}}, \omega)$ \tcp*{计算综合得分}
}
$\hat{\omega} \leftarrow \arg\max_{\omega \in \Omega'(D_{\text{test}})}S(D_{\text{test}}, \omega)$\;
\KwRet{$\hat{\omega}$}
\end{algorithm}

在此基础上,根据损失率 \(\eta(D_{\text{test}})\) 与综合加速比 \(\mathcal{A}(D_{\text{test}})\) 即可评估优化后的时间效率与聚类质量损失情况。

%---------------------------------
% 参考文献(可选)
%---------------------------------
%\bibliographystyle{unsrt}
%\bibliography{references}

\end{document}
